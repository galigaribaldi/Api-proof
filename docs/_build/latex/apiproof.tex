%% Generated by Sphinx.
\def\sphinxdocclass{report}
\documentclass[a4paper,10pt,english]{sphinxmanual}
\ifdefined\pdfpxdimen
   \let\sphinxpxdimen\pdfpxdimen\else\newdimen\sphinxpxdimen
\fi \sphinxpxdimen=.75bp\relax
\ifdefined\pdfimageresolution
    \pdfimageresolution= \numexpr \dimexpr1in\relax/\sphinxpxdimen\relax
\fi
%% let collapsable pdf bookmarks panel have high depth per default
\PassOptionsToPackage{bookmarksdepth=5}{hyperref}

\PassOptionsToPackage{warn}{textcomp}
\usepackage[utf8]{inputenc}
\ifdefined\DeclareUnicodeCharacter
% support both utf8 and utf8x syntaxes
  \ifdefined\DeclareUnicodeCharacterAsOptional
    \def\sphinxDUC#1{\DeclareUnicodeCharacter{"#1}}
  \else
    \let\sphinxDUC\DeclareUnicodeCharacter
  \fi
  \sphinxDUC{00A0}{\nobreakspace}
  \sphinxDUC{2500}{\sphinxunichar{2500}}
  \sphinxDUC{2502}{\sphinxunichar{2502}}
  \sphinxDUC{2514}{\sphinxunichar{2514}}
  \sphinxDUC{251C}{\sphinxunichar{251C}}
  \sphinxDUC{2572}{\textbackslash}
\fi
\usepackage{cmap}
\usepackage[T1]{fontenc}
\usepackage{amsmath,amssymb,amstext}
\usepackage{babel}



\usepackage{tgtermes}
\usepackage{tgheros}
\renewcommand{\ttdefault}{txtt}



\usepackage[Bjarne]{fncychap}
\usepackage{sphinx}

\fvset{fontsize=auto}
\usepackage{geometry}


% Include hyperref last.
\usepackage{hyperref}
% Fix anchor placement for figures with captions.
\usepackage{hypcap}% it must be loaded after hyperref.
% Set up styles of URL: it should be placed after hyperref.
\urlstyle{same}

\addto\captionsenglish{\renewcommand{\contentsname}{Contents:}}

\usepackage{sphinxmessages}
\setcounter{tocdepth}{1}



\title{Api Proof}
\date{Jan 26, 2022}
\release{1.0}
\author{Galileo Garibaldi, Simon Valiterra}
\newcommand{\sphinxlogo}{\vbox{}}
\renewcommand{\releasename}{Release}
\makeindex
\begin{document}

\pagestyle{empty}
\sphinxmaketitle
\pagestyle{plain}
\sphinxtableofcontents
\pagestyle{normal}
\phantomsection\label{\detokenize{index::doc}}

\index{module@\spxentry{module}!configuration@\spxentry{configuration}}\index{configuration@\spxentry{configuration}!module@\spxentry{module}}

\chapter{Configuration}
\label{\detokenize{index:configuration}}
\sphinxAtStartPar
Archivo de configuraciones para cada etapa del proyecto
\index{BaseConfig (class in configuration)@\spxentry{BaseConfig}\spxextra{class in configuration}}

\begin{fulllineitems}
\phantomsection\label{\detokenize{index:configuration.BaseConfig}}\pysigline{\sphinxbfcode{\sphinxupquote{class }}\sphinxcode{\sphinxupquote{configuration.}}\sphinxbfcode{\sphinxupquote{BaseConfig}}}
\sphinxAtStartPar
Configuración de tipo Base en la cual se ponen los siguientes parametros:

\begin{sphinxVerbatim}[commandchars=\\\{\}]
\PYG{o}{|} \PYG{l+m+mf}{1.}\PYG{o}{\PYGZhy{}} \PYG{n}{Secret} \PYG{n}{key}
\PYG{o}{|} \PYG{l+m+mf}{2.}\PYG{o}{\PYGZhy{}} \PYG{n}{Debug}
\PYG{o}{|} \PYG{l+m+mf}{3.}\PYG{o}{\PYGZhy{}} \PYG{n}{Testing}
\PYG{o}{|} \PYG{l+m+mf}{4.}\PYG{o}{\PYGZhy{}} \PYG{n}{SqlAlchemy} \PYG{n}{Database} \PYG{n}{URI}
\PYG{o}{|} \PYG{l+m+mf}{5.}\PYG{o}{\PYGZhy{}} \PYG{n}{SqlAlchemy} \PYG{n}{Track} \PYG{n}{Modifications}
\end{sphinxVerbatim}

\end{fulllineitems}

\index{DevelopmentConfig (class in configuration)@\spxentry{DevelopmentConfig}\spxextra{class in configuration}}

\begin{fulllineitems}
\phantomsection\label{\detokenize{index:configuration.DevelopmentConfig}}\pysigline{\sphinxbfcode{\sphinxupquote{class }}\sphinxcode{\sphinxupquote{configuration.}}\sphinxbfcode{\sphinxupquote{DevelopmentConfig}}}
\sphinxAtStartPar
Configuración de Desarrollo
\begin{description}
\item[{BaseConfig}] \leavevmode{[}class{]}
\sphinxAtStartPar
Configuración base

\end{description}

\end{fulllineitems}

\index{ProductionConfig (class in configuration)@\spxentry{ProductionConfig}\spxextra{class in configuration}}

\begin{fulllineitems}
\phantomsection\label{\detokenize{index:configuration.ProductionConfig}}\pysigline{\sphinxbfcode{\sphinxupquote{class }}\sphinxcode{\sphinxupquote{configuration.}}\sphinxbfcode{\sphinxupquote{ProductionConfig}}}
\sphinxAtStartPar
Configuración de producción
\begin{description}
\item[{BaseConfig}] \leavevmode{[}class{]}
\sphinxAtStartPar
Configuración base

\end{description}

\end{fulllineitems}

\phantomsection\label{\detokenize{index:module-api3.student_api.api_student}}\index{module@\spxentry{module}!api3.student\_api.api\_student@\spxentry{api3.student\_api.api\_student}}\index{api3.student\_api.api\_student@\spxentry{api3.student\_api.api\_student}!module@\spxentry{module}}

\chapter{Api Student}
\label{\detokenize{index:api-student}}
\sphinxAtStartPar
Archivo encargado de definir las rutas y los parámetros a funcionar dentor de cada de
endpoint con los protocolos Web (POST, GET, DELETE, PUT)


\chapter{Notes}
\label{\detokenize{index:notes}}
\sphinxAtStartPar
Ruta en la que se pone la API: \sphinxstylestrong{/api/student/}
\index{ControllerStudent (class in api3.student\_api.api\_student)@\spxentry{ControllerStudent}\spxextra{class in api3.student\_api.api\_student}}

\begin{fulllineitems}
\phantomsection\label{\detokenize{index:api3.student_api.api_student.ControllerStudent}}\pysigline{\sphinxbfcode{\sphinxupquote{class }}\sphinxcode{\sphinxupquote{api3.student\_api.api\_student.}}\sphinxbfcode{\sphinxupquote{ControllerStudent}}}
\sphinxAtStartPar
Clase encargada de definir los protocolos Web Get, Post, Delete, Put.
\begin{description}
\item[{MethodView}] \leavevmode{[}class{]}
\sphinxAtStartPar
Clase heredada de la biblioteca de Flask (flask.views)

\end{description}
\index{delete() (api3.student\_api.api\_student.ControllerStudent method)@\spxentry{delete()}\spxextra{api3.student\_api.api\_student.ControllerStudent method}}

\begin{fulllineitems}
\phantomsection\label{\detokenize{index:api3.student_api.api_student.ControllerStudent.delete}}\pysiglinewithargsret{\sphinxbfcode{\sphinxupquote{delete}}}{\emph{\DUrole{n}{id}}}{}
\sphinxAtStartPar
Método encargado de eliminar registros
\begin{description}
\item[{id}] \leavevmode{[}int{]}
\sphinxAtStartPar
ID del Estudiante a eliminar

\end{description}
\begin{description}
\item[{JSON}] \leavevmode
\sphinxAtStartPar
JSON con la estructura mencionada en \sphinxstylestrong{sendResJson}

\end{description}

\end{fulllineitems}

\index{get() (api3.student\_api.api\_student.ControllerStudent method)@\spxentry{get()}\spxextra{api3.student\_api.api\_student.ControllerStudent method}}

\begin{fulllineitems}
\phantomsection\label{\detokenize{index:api3.student_api.api_student.ControllerStudent.get}}\pysiglinewithargsret{\sphinxbfcode{\sphinxupquote{get}}}{\emph{\DUrole{n}{id}\DUrole{o}{=}\DUrole{default_value}{None}}}{}
\sphinxAtStartPar
Método Get para traer los datos de Estudiantes
\begin{description}
\item[{id}] \leavevmode{[}int, optional{]}
\sphinxAtStartPar
ID del estudiante a encontrar, by default None

\end{description}
\begin{description}
\item[{JSON}] \leavevmode
\sphinxAtStartPar
JSON con la estructura mencionada en \sphinxstylestrong{sendResJson}

\end{description}

\end{fulllineitems}

\index{post() (api3.student\_api.api\_student.ControllerStudent method)@\spxentry{post()}\spxextra{api3.student\_api.api\_student.ControllerStudent method}}

\begin{fulllineitems}
\phantomsection\label{\detokenize{index:api3.student_api.api_student.ControllerStudent.post}}\pysiglinewithargsret{\sphinxbfcode{\sphinxupquote{post}}}{}{}
\sphinxAtStartPar
Método para publicar un nuevo estudiante
\begin{description}
\item[{Se describe el Json que se manda para publicar un estudiante::}] \leavevmode
\sphinxAtStartPar
\$ \{ “age”:2, “name”:”Nombre”, “status”:1 \}

\end{description}
\begin{description}
\item[{JSON}] \leavevmode
\sphinxAtStartPar
JSON con la estructura mencionada en \sphinxstylestrong{sendResJson}

\end{description}

\end{fulllineitems}

\index{put() (api3.student\_api.api\_student.ControllerStudent method)@\spxentry{put()}\spxextra{api3.student\_api.api\_student.ControllerStudent method}}

\begin{fulllineitems}
\phantomsection\label{\detokenize{index:api3.student_api.api_student.ControllerStudent.put}}\pysiglinewithargsret{\sphinxbfcode{\sphinxupquote{put}}}{}{}
\sphinxAtStartPar
Método encargado de modificar los registros
\begin{description}
\item[{JSON}] \leavevmode
\sphinxAtStartPar
JSON con la estructura mencionada en \sphinxstylestrong{sendResJson}

\end{description}

\end{fulllineitems}


\end{fulllineitems}

\phantomsection\label{\detokenize{index:module-api3.student_api.controller}}\index{module@\spxentry{module}!api3.student\_api.controller@\spxentry{api3.student\_api.controller}}\index{api3.student\_api.controller@\spxentry{api3.student\_api.controller}!module@\spxentry{module}}

\chapter{Controller}
\label{\detokenize{index:controller}}
\sphinxAtStartPar
Archivo encargado de hacer las operaciones de DDL en SQL, es decir, agregar, seleccionar y eliminar
\index{convertToJson() (in module api3.student\_api.controller)@\spxentry{convertToJson()}\spxextra{in module api3.student\_api.controller}}

\begin{fulllineitems}
\phantomsection\label{\detokenize{index:api3.student_api.controller.convertToJson}}\pysiglinewithargsret{\sphinxcode{\sphinxupquote{api3.student\_api.controller.}}\sphinxbfcode{\sphinxupquote{convertToJson}}}{\emph{\DUrole{n}{student}\DUrole{p}{:} \DUrole{n}{{\hyperref[\detokenize{index:api3.student_api.models.Student.Student}]{\sphinxcrossref{api3.student\_api.models.Student.Student}}}}}}{}
\sphinxAtStartPar
Convertir el objeto de tipo \sphinxstylestrong{Student}
\begin{description}
\item[{student}] \leavevmode{[}Student{]}
\sphinxAtStartPar
Objeto de tipo \sphinxstylestrong{Student}, el cual tiene como parámetros: id, name, age, state

\end{description}
\begin{description}
\item[{JSON}] \leavevmode
\sphinxAtStartPar
json con el formato de los tipos de dato

\end{description}

\end{fulllineitems}

\index{delete\_student() (in module api3.student\_api.controller)@\spxentry{delete\_student()}\spxextra{in module api3.student\_api.controller}}

\begin{fulllineitems}
\phantomsection\label{\detokenize{index:api3.student_api.controller.delete_student}}\pysiglinewithargsret{\sphinxcode{\sphinxupquote{api3.student\_api.controller.}}\sphinxbfcode{\sphinxupquote{delete\_student}}}{\emph{\DUrole{n}{id}}}{}
\sphinxAtStartPar
Eliminar un registro de el estudiante por su ID
\begin{description}
\item[{id}] \leavevmode{[}int{]}
\sphinxAtStartPar
ID del estudiante (PK)

\end{description}
\begin{description}
\item[{Json}] \leavevmode
\sphinxAtStartPar
json con el formato de los tipos de dato

\end{description}

\end{fulllineitems}

\index{insert\_student() (in module api3.student\_api.controller)@\spxentry{insert\_student()}\spxextra{in module api3.student\_api.controller}}

\begin{fulllineitems}
\phantomsection\label{\detokenize{index:api3.student_api.controller.insert_student}}\pysiglinewithargsret{\sphinxcode{\sphinxupquote{api3.student\_api.controller.}}\sphinxbfcode{\sphinxupquote{insert\_student}}}{\emph{\DUrole{n}{json\_data}}}{}
\sphinxAtStartPar
Insertar datos al json de Estudiante
\begin{description}
\item[{json\_data}] \leavevmode{[}json{]}
\sphinxAtStartPar
Json con el formato: \{ “age”:2, “name”:”Nombre”, “status”:1 \}

\end{description}
\begin{description}
\item[{Json}] \leavevmode
\sphinxAtStartPar
json con el formato de los tipos de dato

\end{description}

\end{fulllineitems}

\index{select\_all\_student() (in module api3.student\_api.controller)@\spxentry{select\_all\_student()}\spxextra{in module api3.student\_api.controller}}

\begin{fulllineitems}
\phantomsection\label{\detokenize{index:api3.student_api.controller.select_all_student}}\pysiglinewithargsret{\sphinxcode{\sphinxupquote{api3.student\_api.controller.}}\sphinxbfcode{\sphinxupquote{select\_all\_student}}}{}{}
\sphinxAtStartPar
Seleccionar todos los registros de los estudiantes
\begin{description}
\item[{Json}] \leavevmode
\sphinxAtStartPar
json con el formato de los tipos de dato

\end{description}

\end{fulllineitems}

\index{select\_student\_by\_id() (in module api3.student\_api.controller)@\spxentry{select\_student\_by\_id()}\spxextra{in module api3.student\_api.controller}}

\begin{fulllineitems}
\phantomsection\label{\detokenize{index:api3.student_api.controller.select_student_by_id}}\pysiglinewithargsret{\sphinxcode{\sphinxupquote{api3.student\_api.controller.}}\sphinxbfcode{\sphinxupquote{select\_student\_by\_id}}}{\emph{\DUrole{n}{id}}}{}
\sphinxAtStartPar
Seleccionar registros de estudiantes por su ID
\begin{description}
\item[{id}] \leavevmode{[}int{]}
\sphinxAtStartPar
Id del estudiante (PK)

\end{description}
\begin{description}
\item[{Json}] \leavevmode
\sphinxAtStartPar
json con el formato de los tipos de dato

\end{description}

\end{fulllineitems}

\phantomsection\label{\detokenize{index:module-api3.helper.convert_data}}\index{module@\spxentry{module}!api3.helper.convert\_data@\spxentry{api3.helper.convert\_data}}\index{api3.helper.convert\_data@\spxentry{api3.helper.convert\_data}!module@\spxentry{module}}

\chapter{Convert data}
\label{\detokenize{index:convert-data}}\begin{description}
\item[{Archivo encargado de darle un formato establecido de json, el cual se describe a continuación::}] \leavevmode
\begin{DUlineblock}{0em}
\item[] \{
\item[]
\begin{DUlineblock}{\DUlineblockindent}
\item[] ‘code’: Codigo error o Aprobatorio
\item[] ‘data’: lista de valores con el json de los datos
\item[] ‘message’: Sucess \sphinxhyphen{} Fail
\end{DUlineblock}
\item[] \} 
\end{DUlineblock}

\end{description}
\index{sendResJson() (in module api3.helper.convert\_data)@\spxentry{sendResJson()}\spxextra{in module api3.helper.convert\_data}}

\begin{fulllineitems}
\phantomsection\label{\detokenize{index:api3.helper.convert_data.sendResJson}}\pysiglinewithargsret{\sphinxcode{\sphinxupquote{api3.helper.convert\_data.}}\sphinxbfcode{\sphinxupquote{sendResJson}}}{\emph{\DUrole{n}{data}}, \emph{\DUrole{n}{code}}}{}
\sphinxAtStartPar
Método encargada de Mapear los datos a un json estandarizado
\begin{description}
\item[{data}] \leavevmode{[}list{]}
\sphinxAtStartPar
lista de datos recibidos de una consulta, ésta puede estar vacía

\item[{code}] \leavevmode{[}int{]}
\begin{DUlineblock}{0em}
\item[] 200 \sphinxhyphen{}\textgreater{} Satisfactorio.
\item[] 404 \sphinxhyphen{}\textgreater{} No Encontrado.
\item[] Cualquier otro, dará fail
\end{DUlineblock}

\end{description}
\begin{description}
\item[{JSON}] \leavevmode
\sphinxAtStartPar
Json de respuesta

\end{description}

\end{fulllineitems}

\phantomsection\label{\detokenize{index:module-api3.student_api.models.Student}}\index{module@\spxentry{module}!api3.student\_api.models.Student@\spxentry{api3.student\_api.models.Student}}\index{api3.student\_api.models.Student@\spxentry{api3.student\_api.models.Student}!module@\spxentry{module}}

\chapter{Student}
\label{\detokenize{index:student}}
\sphinxAtStartPar
Modelo de la tabla Student, la cual se mapea a la base de datos con la ayuda de la bilbioteca
SQLAlchemy
\index{Student (class in api3.student\_api.models.Student)@\spxentry{Student}\spxextra{class in api3.student\_api.models.Student}}

\begin{fulllineitems}
\phantomsection\label{\detokenize{index:api3.student_api.models.Student.Student}}\pysiglinewithargsret{\sphinxbfcode{\sphinxupquote{class }}\sphinxcode{\sphinxupquote{api3.student\_api.models.Student.}}\sphinxbfcode{\sphinxupquote{Student}}}{\emph{\DUrole{n}{name}}, \emph{\DUrole{n}{age}}, \emph{\DUrole{n}{status}}}{}
\sphinxAtStartPar
Mapeado de Student
\begin{description}
\item[{db}] \leavevmode{[}Class{]}
\sphinxAtStartPar
Clase heredada de SQLAlchemy, \sphinxstylestrong{NO MODIFICABLE}.
Se describen los parámetros que tiene la tabla Student:
\begin{quote}

\begin{DUlineblock}{0em}
\item[] 1.\sphinxhyphen{} ID: Entero (Llave primaria)
\item[] 2.\sphinxhyphen{} Name: Cadena
\item[] 3.\sphinxhyphen{} Age: Entero 
\item[] 4.\sphinxhyphen{} Status: Bool
\end{DUlineblock}
\end{quote}

\end{description}
\begin{description}
\item[{tuple}] \leavevmode
\sphinxAtStartPar
Tupla con el siguiente formato: “Task Id, Name, Duracion”

\end{description}

\end{fulllineitems}



\chapter{Indices and tables}
\label{\detokenize{index:indices-and-tables}}\begin{itemize}
\item {} 
\sphinxAtStartPar
\DUrole{xref,std,std-ref}{genindex}

\item {} 
\sphinxAtStartPar
\DUrole{xref,std,std-ref}{modindex}

\item {} 
\sphinxAtStartPar
\DUrole{xref,std,std-ref}{search}

\end{itemize}


\renewcommand{\indexname}{Python Module Index}
\begin{sphinxtheindex}
\let\bigletter\sphinxstyleindexlettergroup
\bigletter{a}
\item\relax\sphinxstyleindexentry{api3.helper.convert\_data}\sphinxstyleindexpageref{index:\detokenize{module-api3.helper.convert_data}}
\item\relax\sphinxstyleindexentry{api3.student\_api.api\_student}\sphinxstyleindexpageref{index:\detokenize{module-api3.student_api.api_student}}
\item\relax\sphinxstyleindexentry{api3.student\_api.controller}\sphinxstyleindexpageref{index:\detokenize{module-api3.student_api.controller}}
\item\relax\sphinxstyleindexentry{api3.student\_api.models.Student}\sphinxstyleindexpageref{index:\detokenize{module-api3.student_api.models.Student}}
\indexspace
\bigletter{c}
\item\relax\sphinxstyleindexentry{configuration}\sphinxstyleindexpageref{index:\detokenize{module-configuration}}
\end{sphinxtheindex}

\renewcommand{\indexname}{Index}
\printindex
\end{document}